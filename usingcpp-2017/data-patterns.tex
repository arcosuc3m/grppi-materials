\subsection{Data patterns}

\begin{frame}[t]{Patterns on data sets}
\begin{itemize}
  \item A \textgood{data pattern} performs an operation on one 
        or more data sets that are already in memory.

  \vfill 
  \item \textmark{Input}:
    \begin{itemize}
      \item One or more data sets.
      \item Operations.
    \end{itemize}

  \vfill 
  \item \textmark{Output}:
    \begin{itemize}
      \item A data set (\cppid{map}, \cppid{stencil}).
      \item A single value (\cppid{reduce}, \cppid{map/reduce}).
    \end{itemize}
\end{itemize}
\end{frame}

\begin{frame}[t]{Data sets}
\begin{itemize}
  \item Data sets are assumed to be value sequences with known size.
    \begin{itemize}

      \vfill
      \item $S = x_1, x_2, \ldots , x_N$

      \vfill
      \item \textgood{Unidimensional patterns}: Take a single sequence.
        \begin{itemize}
          \item Specified by pair of iterators.
        \end{itemize}

      \vfill
      \item \textgood{Multidemensional patterns}: Take start of extra sequences.
        \begin{itemize}
          \item Size derived from first sequence.
        \end{itemize}

      \vfill
      \item \textgood{Output}: Take start of output sequence.
        \begin{itemize}
          \item Size derived from first sequence.
        \end{itemize}
    \end{itemize}
\end{itemize}
\end{frame}

\begin{frame}[t]{Maps on data sequences}
\begin{itemize}
  \item A \textgood{map} pattern applies an operation to every element in a
        a data set, generating a new data set.
  \vfill
  \item \textgood{Unidimensional}:
    \begin{itemize}
      \item $S = x_1, x_2, \ldots x_n$
      \item $map(S,f)$
        \begin{itemize} 
          \item $f(x_1), f(x_2), \ldots, f(x_n)$
        \end{itemize}
    \end{itemize}

  \vfill
  \item \textgood{Multidimensional}:
    \begin{itemize}
      \item $S_k = x^k_1, x^k_2, \ldots, x^k_n$
      \item $map(S_1, S_2, \ldots, S_m, f)$
        \begin{itemize}
          \item $f(x^1_1, x^2_1, \ldots, x^m_1), f(x^2_1, x^2_2, \ldots, x^m_2), \ldots$
        \end{itemize}
    \end{itemize}
\end{itemize}
\end{frame}

\begin{frame}[t,fragile]{Single sequences mapping}
\begin{block}{Double all elements in sequence}
\begin{lstlisting}
template <typename Execution>
std::vector<double> double_elements(const Execution & ex, 
                                    const std::vector<double> & v) 
{
  std::vector<double> res(v.size());

  grppi::map(ex, v.begin(), v.end(), res.begin(), 
    [](double x) { return 2*x; });

  return res;
}
\end{lstlisting}
\end{block}
\end{frame}

\begin{frame}[t,fragile]{Multiple sequences mapping}
\begin{block}{Add two vectors}
\begin{lstlisting}
template <typename Execution>
std::vector<double> add_vectors(const Execution & ex, 
                                const std::vector<double> & v1,
                                const std::vector<double> & v2) 
{
  auto size = std::min(v1.size(), v2.size());
  std::vector<double> res(size);

  grppi::map(ex, v1.begin(), v1.end(), res.begin(),
    [](double x, double y) { return x+y; },
    v2.begin());

  return res;
}
\end{lstlisting}
\end{block}
\end{frame}

\begin{frame}[t,fragile]{Multiple sequences mapping}
\begin{block}{Add three vectors}
\begin{lstlisting}
template <typename Execution>
std::vector<double> add_vectors(const Execution & ex, 
                                const std::vector<double> & v1,
                                const std::vector<double> & v2,
                                const std::vector<double> & v3) 
{
  auto size = std::min(v1.size(), v2.size());
  std::vector<double> res(size);

  grppi::map(ex, v1.begin(), v1.end(), res.begin(),
    [](double x, double y, double z) { return x+y+z; },
    v2.begin(), v3.begin());

  return res;
}
\end{lstlisting}
\end{block}
\end{frame}

\begin{frame}[t,fragile]{Heterogeneous mapping}
\begin{itemize}
  \item The result can be from a different type.
\end{itemize}
\begin{block}{Complex vector from real and imaginary vectors}
\begin{lstlisting}
template <typename Execution>
std::vector<complex<double>> create_cplx(const Execution & ex,
                                         const std::vector<double> & re,
                                         const std::vector<double> & im)
{
  auto size = std::min(re.size(), im.size());
  std::vector<complex<double>> res(size);

  grppi::map(ex, re.begin(), re.end(), res.begin(),
    [](double r, double i) -> complex<double> { return {r,i}; }
    im.begin());

  return res;
}
\end{lstlisting}
\end{block}
\end{frame}

\begin{frame}[t]{Reductions on data sequences}
\begin{itemize}
  \item A \textgood{reduce} pattern combines all values in a data set
        using a binary combination operation.
  \vfill

\begin{tikzpicture}
\tikzset{
  label/.style={text centered, text=orange,font=\footnotesize,minimum width=1cm} ,
  transform/.style={rectangle,rounded corners,draw=black,fill=green!50,text=white,thick, text centered, font=\tiny, minimum width=0.75cm,minimum height=0.5cm},
  item/.style={rectangle,draw=black,fill=orange!70,text=white,thick, text centered, font=\tiny, minimum width=0.75cm},
  result/.style={rectangle,draw=black,fill=blue!70,text=white,thick, text centered, font=\tiny, minimum width=0.75cm},
  arrow/.style={->,thick,draw=black,font=\tiny},
}
\node[item,fill=orange!20,text=black] (item0) {id};
\node[item,right=1cm of item1] (item1) {};
\node[item,right=0cm of item1] (item2) {};
\node[item,right=0cm of item2] (item3) {};
\node[item,right=0cm of item3] (item4) {};
\node[item,right=0cm of item4] (item5) {};
%
\node[transform,below=0.25cm of item1,minimum width=0.5cm] (map1) {};
\node[transform,below=0.25cm of item2,minimum width=0.5cm] (map2) {};
\node[transform,below=0.25cm of item3,minimum width=0.5cm] (map3) {};
\node[transform,below=0.25cm of item4,minimum width=0.5cm] (map4) {};
\node[transform,below=0.25cm of item5,minimum width=0.5cm] (map5) {};
%
\node[result,below=0.5cm of map5] (result) {result};
%
\path[arrow](item0) -- (map1);
\path[arrow](item1) -- (map1);
\path[arrow](map1) -- (map2);
\path[arrow](item2) -- (map2);
\path[arrow](map2) -- (map3);
\path[arrow](item3) -- (map3);
\path[arrow](map3) -- (map4);
\path[arrow](item4) -- (map4);
\path[arrow](map4) -- (map5);
\path[arrow](item5) -- (map5);
%
\path[arrow](map5) -- (result);
\end{tikzpicture}


%  \item Given:
%    \begin{itemize}
%      \item A sequence $x_1, x_2, \ldots, x_N \in T$.
%      \item An identity value $id \in I$.
%      \item A combine operation $c : I \times T \mapsto I$
%        \begin{itemize}
%          \item $c(c(x,y),z) \equiv c(x,c(y,z))$
%          \item $c(id,x) = \bar{x}$, where $\bar{x}$ is the value of $x$ in $I$.
%          \item $c(id,c(id,x)) = c(id,x)$
%          \item $c(c(c(id,x),y),c(c(id,z),t)) = c(c(c(c(id,x),y),z),t)$
%        \end{itemize}
%    \end{itemize}
%  \vfill
%  \item It generates the value:
%    \begin{itemize}
%      \item $c(\ldots c(c(id,x_1), x_2) \ldots, x_N)$
%    \end{itemize}
\end{itemize}
\end{frame}

\begin{frame}[t,fragile]{Homogeneous reduction}
\begin{block}{Add a sequence of values}
\begin{lstlisting}
template <typename Execution>
double add_sequence(const Execution & ex, const vector<double> & v)
{
  return grppi::reduce(ex, v, 0.0,
    [](double x, double y) { return x+y; });
}
\end{lstlisting}
\end{block}
\end{frame}

\begin{frame}[t,fragile]{Heterogeneous reduction}
\begin{block}{Add areas of shapes}
\begin{lstlisting}
template <typename Execution>
int add_areas(const Execution & ex, const std::vector<shape> & shapes)
{
  return grppi::reduce(ex, shapes, 0.0,
    [](double a, const auto & s) { 
        return a + s.area()(); 
    });
}
\end{lstlisting}
\end{block}
\begin{itemize}
  \item \textbad{Note}: Better expressed as a map reduce.
\end{itemize}
\end{frame}

\subsection{Map/reduce pattern}

\begin{frame}[t]{Map/reduce pattern}
\begin{itemize}
  \item A \textgood{map/reduce} pattern combines a \textmark{map} pattern and
        a \textmark{reduce} pattern into a single pattern.
    \begin{enumerate}
      \item One or more data sets are \textmark{mapped} applying a transformation operation.
      \item The results are combined by a \textmark{reduction} operation.
    \end{enumerate}
  \vfill
  \item A \textmark{map/reduce} could be also expressed by the composition of a
        \textmark{map} and a \textmark{reduce}. 
    \begin{itemize}
      \item However, \textmark{map/reduce} may potentially fuse both stages, 
            allowing for extra optimizations.
    \end{itemize}
\end{itemize}
\end{frame}

\begin{frame}[t]{Map/reduce with single data set}
\begin{itemize}
  \item A \textgood{map/reduce} on a single input sequence producing a value.
  \vfill\pause
  \item Given:
    \begin{itemize}
      \item A sequence $x_1, x_2, \ldots x_N \in T$
      \item A mapping function $m : T \mapsto R$
      \item A reduction identity value $id \in I$.
      \item A combine operation $c : I \times R \mapsto I$
    \end{itemize}
  \vfill\pause
  \item It generates a value reducing the mapping:
    \begin{itemize}
      \item $c(c(c(id,m_1),m_2), \ldots, m_M)$
      \item Where $m_k = m(x_k)$
    \end{itemize}
\end{itemize}
\end{frame}

\begin{frame}[t,fragile]{Single sequence map/reduce}
\begin{block}{Sum of squares}
\begin{lstlisting}
template <typename Execution>
double sum_squares(const Execution & ex, const std::vector<double> & v)
{
  return grppi::map_reduce(ex, v.begin(), v.end(), 0.0,
    [](double x) { return x*x; }
    [](double x, double y) { return x+y; }
  );
}
\end{lstlisting}
\end{block}
\end{frame}

\begin{frame}[t]{Map/reduce in multiple data sets}
\begin{itemize}
  \item A \textgood{map/reduce} on multiple input sequences producing a single value.
  \vfill\pause
  \item Given:
    \begin{itemize}
      \item A sequence $x^1_1, x^1_2, \ldots x^1_N \in T_1$
      \item A sequence $x^2_1, x^2_2, \ldots x^2_N \in T_2$
      \item \ldots
      \item A sequence $x^M_1, x^M_2, \ldots x^M_N \in T_M$
      \item A mapping function $m : T_1 \times T_2 \times \ldots \times T_M \mapsto R$
      \item A reduction identity value $id \in I$.
      \item A combine operation $c : I \times R \mapsto I$
    \end{itemize}
  \vfill\pause
  \item It generates a value reducing the mapping:
    \begin{itemize}
      \item $c(c(c(id,m_1),m_2), \ldots, m_M)$
      \item Where $m_k = m(x^k_1, x^k_2, \ldots, x^k_N)$
    \end{itemize}
\end{itemize}
\end{frame}

\begin{frame}[t,fragile]{Map/reduce on two data sets}
\begin{block}{Scalar product}
\begin{lstlisting}
template <typename Execution>
double scalar_product(const Execution & ex,
                      const std::vector<double> & v1,
                      const std::vector<double> & v2)
{
  return grppi::map_reduce(ex, begin(v1), end(v1), 0.0,
    [](double x, double y) { return x*y; },
    [](double x, double y) { return x+y; },
    v2.begin());
}
\end{lstlisting}
\end{block}
\end{frame}

\begin{frame}[t,fragile]{Cannonical map/reduce}
\begin{itemize}
  \item Given a sequence of words, produce a \emph{<key,value>} container where:
    \begin{itemize}
      \item The key is the word.
      \item The value is the number of occurrences of that word.
    \end{itemize}
\end{itemize}
\vfill\pause
\begin{block}{Word frequencies}
\begin{lstlisting}
template <typename Execution>
auto word_freq(const Execution & ex, const std::vector<std::string> & words)
{
  using namespace std;
  using dictionary = std::map<string,int>;
  return grppi::map_reduce(ex, words.begin(), words.end(), dictionary{},
    [](string w) -> dictionary { return {w,1}; }
    [](dictionary & lhs, const dictionary & rhs) -> dictionary {
      for (auto & entry : rhs) { lhs[entry.first] += entry.second; }
      return lhs;
    });
}
\end{lstlisting}
\end{block}
\end{frame}

\begin{frame}[t]{Stencil pattern}
\begin{itemize}
  \item A \textgood{stencil} pattern applies a transformation to every 
        element in one or multiple data sets, generating a new data set as an output
    \begin{itemize}
      \item The transformation is function of a data item and its \emph{neighbourhood}.
    \end{itemize}
\end{itemize}
\vfill
\begin{center}
\begin{tikzpicture}
\tikzset{
  label/.style={text centered, text=orange,font=\footnotesize,minimum width=1cm} ,
  transform/.style={rectangle,rounded corners,draw=black,fill=green!50,text=white,thick, text centered, font=\tiny, minimum width=0.75cm,minimum height=0.5cm},
  item/.style={rectangle,draw=black,fill=orange!70,text=white,thick, text centered, font=\tiny, minimum width=0.75cm},
  result/.style={rectangle,draw=black,fill=blue!70,text=white,thick, text centered, font=\tiny, minimum width=0.75cm},
  arrow/.style={->,thick,draw=black,font=\tiny},
}  
\node[item] (item1) {};
\node[item,right=0cm of item1] (item2) {};
\node[item,right=0cm of item2] (item3) {};
\node[item,right=0cm of item3] (item4) {};
\node[item,right=0cm of item4] (item5) {};
%
\node[result,below=2cm of item1] (ritem1) {};
\node[result,right=0cm of ritem1] (ritem2) {};
\node[result,right=0cm of ritem2] (ritem3) {};
\node[result,right=0cm of ritem3] (ritem4) {};
\node[result,right=0cm of ritem4] (ritem5) {};
%
\node[transform,below=0.75cm of item1,minimum width=0.5cm] (map1) {};
\node[transform,below=0.75cm of item2,minimum width=0.5cm] (map2) {};
\node[transform,below=0.75cm of item3,minimum width=0.5cm] (map3) {};
\node[transform,below=0.75cm of item4,minimum width=0.5cm] (map4) {};
\node[transform,below=0.75cm of item5,minimum width=0.5cm] (map5) {};
%
\draw[arrow] (item1) -- (map1);
\draw[arrow] (item2) -- (map1);
%
\draw[arrow] (item1) -- (map2);
\draw[arrow] (item2) -- (map2);
\draw[arrow] (item3) -- (map2);
%
\draw[arrow] (item2) -- (map3);
\draw[arrow] (item3) -- (map3);
\draw[arrow] (item4) -- (map3);
%
\draw[arrow] (item3) -- (map4);
\draw[arrow] (item4) -- (map4);
\draw[arrow] (item5) -- (map4);
%
\draw[arrow] (item4) -- (map5);
\draw[arrow] (item5) -- (map5);
%
\path[arrow](map1) -- (ritem1);
\path[arrow](map2) -- (ritem2);
\path[arrow](map3) -- (ritem3);
\path[arrow](map4) -- (ritem4);
\path[arrow](map5) -- (ritem5);
\end{tikzpicture}
\end{center}
\end{frame}

\begin{frame}[t,fragile]{Single sequence stencil}
\begin{block}{Neighbour average}
\begin{lstlisting}
template <typename Execution>
std::vector<double> neib_avg(const Execution & ex, const std::vector<double> & v)
{
  std::vector<double> res(v.size());
  grppi::stencil(ex, v.begin(), v.end(), 
    [](auto it, auto n) {
      return *it + accumulate(begin(n), end(n)); 
    },
    [&](auto it) {
      vector<double> r;
      if (it!=begin(v)) r.push_back(*prev(it));
      if (distance(it,end(end))>1) r.push_back(*next(it));
      return r;
    });
  return res;
}
\end{lstlisting}
\end{block}
\end{frame}


