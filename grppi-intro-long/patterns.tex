\begin{frame}[t]{Software design}

\begin{quote}
There are two ways of constructing a software design:\\ 
\vspace{1em}
\pause
One way is\\
\pause
to make it \textgood{so simple} that there are \textmark{obviously no deficiencies},\\
\pause
\vspace{.5em}
and the other way is\\
\pause
to make it \textgood{so complicated} that there are \textmark{no obvious deficiencies}.\\ 
\vspace{1em}
\pause
The \textmark{first method} is \textbad{far more difficult}. 
\end{quote}
\hfill C.A.R Hoare
\end{frame}

\begin{frame}[t,fragile]{Adding two vectors}
\begin{block}{Traditional way}
\lstinputlisting[firstline=6,lastline=15]{ej/src/addvec/classic.cpp}
\end{block}
\pause
\begin{itemize}
  \item Adds additional constraints.
    \begin{itemize}
      \item Traversing in-order.
    \end{itemize}
  \item Potential mistakes.
    \begin{itemize}
      \item \cppid{i<v1.size()} versus \cppid{i<=v1.size()}.
    \end{itemize}
\end{itemize}
\end{frame}

\begin{frame}[t,fragile]{Adding two vectors}
\begin{block}{The STL way (C++98/03)}
\lstinputlisting[firstline=6,lastline=15]{ej/src/addvec/stl.cpp}
\end{block}
\pause
\begin{itemize}
  \item Minimize off-by-one mistakes.
  \item Type specific optimizations.
\end{itemize}
\end{frame}

\begin{frame}[t,fragile]{Adding two vectors}
\begin{block}{The STL way (C++14)}
\lstinputlisting[firstline=6,lastline=15]{ej/src/addvec/stl11.cpp}
\end{block}
\pause
\begin{itemize}
  \item Use (possibly generic) lambdas .
\end{itemize}
\end{frame}

\begin{frame}[t]{A brief history of patterns}
\begin{itemize}
\item From building and architecture (Cristopher Alexander):
\begin{itemize}
  \item \textbf{1977}: A Pattern Language: Towns, Buildings, Construction.
  \item \textbf{1979}: The timeless way of buildings.
\end{itemize}
\pause\vfill
\item To software design (Gamma et al.):
\begin{itemize}
  \item \textbf{1993}: Design Patterns: abstraction and reuse of object oriented design. ECOOP.
  \item \textbf{1995}: Design Patterns. Elements of Reusable Object-Oriented Software.
\end{itemize}
\pause\vfill
\item To parallel programming (McCool, Reinders, Robinson):
\begin{itemize}
  \item \textbf{2012}: Structured Parallel Programming: Patterns for Efficient Computation.
\end{itemize}
\end{itemize}
\end{frame}

\begin{frame}[t]{Parallel Patterns}
\begin{itemize}
  \item Levels of patterns:
    \begin{itemize}
      \item Software Architecture Patterns.
      \item Design Patterns.
      \item Algorithm Strategy Patterns.
      \item Implementation patterns.
    \end{itemize}

  \vfill\pause
  \item \textmark{Algorithm Strategy Pattern} or \textmark{Algorithmic Skeleton}
    \begin{itemize}
      \item A way to codify best practices in \textmark{parallel programming} 
            in a \textgood{reusable way}.
    \end{itemize}

  \vfill\pause
  \item A single \textmark{Parallel Pattern} may have \textgood{different realizations}
        in \textgood{different programming models}.
  
\end{itemize}
\end{frame}

\begin{frame}[t,fragile]{Example: map pattern}
\begin{itemize}
  \item Apply a single \textmark{elemental operation} to different \textmark{data items}.
\end{itemize}
\begin{columns}

\column{.4\textwidth}
\begin{block}{SAXPY: Sequential}
\begin{lstlisting}
for (size_t i=0; i<n; ++i) {
  y[i] = a * x[i] + y[i];
}
\end{lstlisting}
\end{block}

\pause

\begin{block}{SAXPY: OpenMP}
\begin{lstlisting}[morekeywords={[1]{omp,parallel}}]
#pragma omp parallel for
for (size_t i=0; i<n; ++i) {
  y[i] = a * x[i] + y[i];
}
\end{lstlisting}
\end{block}

\pause

\column{.6\textwidth}
\begin{block}{SAXPY: TBB}
\begin{lstlisting}
tbb::parallel_for(tbb::blocked_range<int>{0,n},
  [&](tbb::blocked_range<int> r) {
    for (auto i : r) {
      y[i] = a * x[i] + y[i];
    }
  });
\end{lstlisting}
\end{block}
\end{columns}
\end{frame}
