\begin{frame}[t]{Software design}

\begin{quote}
There are two ways of constructing a software design:\\ 
\vspace{1em}
\pause
One way is\\
\pause
to make it \textgood{so simple} that there are \textmark{obviously no deficiencies},\\
\pause
\vspace{.5em}
and the other way is\\
\pause
to make it \textgood{so complicated} that there are \textmark{no obvious deficiencies}.\\ 
\vspace{1em}
\pause
The \textmark{first method} is \textbad{far more difficult}. 
\end{quote}
\hfill C.A.R Hoare
\end{frame}

\begin{frame}[t,fragile]{Adding two vectors}
\begin{block}{Traditional way}
\lstinputlisting[firstline=6,lastline=15]{ej/src/addvec/classic.cpp}
\end{block}
\pause
\begin{itemize}
  \item Adds additional constraints.
    \begin{itemize}
      \item Traversing in-order.
    \end{itemize}
  \item Potential mistakes.
    \begin{itemize}
      \item \cppid{i<v1.size()} versus \cppid{i<=v1.size()}.
    \end{itemize}
\end{itemize}
\end{frame}

\begin{frame}[t,fragile]{Adding two vectors}
\begin{block}{The STL way}
\lstinputlisting[firstline=6,lastline=16]{ej/src/addvec/stl.cpp}
\end{block}
\pause
\begin{itemize}
  \item Does not add additional constraints (ordering).
  \item Less error prone.
\end{itemize}
\end{frame}

\begin{frame}[t]{A brief history of patterns}
\begin{itemize}
\item From building and architecture (Cristopher Alexander):
\begin{itemize}
  \item \textbf{1977}: A Pattern Language: Towns, Buildings, Construction.
  \item \textbf{1979}: The timeless way of buildings.
\end{itemize}
\vfill
\item To software design (Gamma et al.):
\begin{itemize}
  \item \textbf{1993}: Design Patterns: abstraction and reuse of object oriented design. ECOOP.
  \item \textbf{1995}: Design Patterns. Elements of Reusable Object-Oriented Software.
\end{itemize}
\vfill
\item To parallel programming (McCool, Reinders, Robinson):
\begin{itemize}
  \item \textbf{2012}: Structured Parallel Programming: Patterns for Efficient Computation.
\end{itemize}
\end{itemize}
\end{frame}
