\begin{frame}[t]{Stencil pattern}
\begin{itemize}
  \item A \textgood{stencil} pattern applies a transformation to every 
        element in one or multiple data sets, generating a new data set as an output
    \begin{itemize}
      \item The transformation is function of a data item and its \emph{neighbourhood}.
    \end{itemize}
\end{itemize}
\vfill
\begin{center}
\begin{tikzpicture}
\tikzset{
  label/.style={text centered, text=orange,font=\footnotesize,minimum width=1cm} ,
  transform/.style={rectangle,rounded corners,draw=black,fill=green!50,text=white,thick, text centered, font=\tiny, minimum width=0.75cm,minimum height=0.5cm},
  item/.style={rectangle,draw=black,fill=orange!70,text=white,thick, text centered, font=\tiny, minimum width=0.75cm},
  result/.style={rectangle,draw=black,fill=blue!70,text=white,thick, text centered, font=\tiny, minimum width=0.75cm},
  arrow/.style={->,thick,draw=black,font=\tiny},
}  
\node[item] (item1) {};
\node[item,right=0cm of item1] (item2) {};
\node[item,right=0cm of item2] (item3) {};
\node[item,right=0cm of item3] (item4) {};
\node[item,right=0cm of item4] (item5) {};
%
\node[result,below=2cm of item1] (ritem1) {};
\node[result,right=0cm of ritem1] (ritem2) {};
\node[result,right=0cm of ritem2] (ritem3) {};
\node[result,right=0cm of ritem3] (ritem4) {};
\node[result,right=0cm of ritem4] (ritem5) {};
%
\node[transform,below=0.75cm of item1,minimum width=0.5cm] (map1) {};
\node[transform,below=0.75cm of item2,minimum width=0.5cm] (map2) {};
\node[transform,below=0.75cm of item3,minimum width=0.5cm] (map3) {};
\node[transform,below=0.75cm of item4,minimum width=0.5cm] (map4) {};
\node[transform,below=0.75cm of item5,minimum width=0.5cm] (map5) {};
%
\draw[arrow] (item1) -- (map1);
\draw[arrow] (item2) -- (map1);
%
\draw[arrow] (item1) -- (map2);
\draw[arrow] (item2) -- (map2);
\draw[arrow] (item3) -- (map2);
%
\draw[arrow] (item2) -- (map3);
\draw[arrow] (item3) -- (map3);
\draw[arrow] (item4) -- (map3);
%
\draw[arrow] (item3) -- (map4);
\draw[arrow] (item4) -- (map4);
\draw[arrow] (item5) -- (map4);
%
\draw[arrow] (item4) -- (map5);
\draw[arrow] (item5) -- (map5);
%
\path[arrow](map1) -- (ritem1);
\path[arrow](map2) -- (ritem2);
\path[arrow](map3) -- (ritem3);
\path[arrow](map4) -- (ritem4);
\path[arrow](map5) -- (ritem5);
\end{tikzpicture}
\end{center}
\end{frame}

\begin{frame}[t,fragile]{Single sequence stencil}
\begin{block}{Neighbour average}
\begin{lstlisting}
template <typename Execution>
std::vector<double> neib_avg(const Execution & ex, const std::vector<double> & v)
{
  std::vector<double> res(v.size());
  grppi::stencil(ex, v.begin(), v.end(), 
    [](auto it, auto n) {
      return *it + accumulate(begin(n), end(n)); 
    },
    [&](auto it) {
      vector<double> r;
      if (it!=begin(v)) r.push_back(*prev(it));
      if (distance(it,end(end))>1) r.push_back(*next(it));
      return r;
    });
  return res;
}
\end{lstlisting}
\end{block}
\end{frame}

