\begin{frame}[t]{Reductions on data sequences}
\begin{itemize}
  \item A \textgood{reduce} pattern combines all values in a data set
        using a binary combination operation.
  \vfill

\begin{tikzpicture}
\tikzset{
  label/.style={text centered, text=orange,font=\footnotesize,minimum width=1cm} ,
  transform/.style={rectangle,rounded corners,draw=black,fill=green!50,text=white,thick, text centered, font=\tiny, minimum width=0.75cm,minimum height=0.5cm},
  item/.style={rectangle,draw=black,fill=orange!70,text=white,thick, text centered, font=\tiny, minimum width=0.75cm},
  result/.style={rectangle,draw=black,fill=blue!70,text=white,thick, text centered, font=\tiny, minimum width=0.75cm},
  arrow/.style={->,thick,draw=black,font=\tiny},
}
\node[item,fill=orange!20,text=black] (item0) {id};
\node[item,right=1cm of item1] (item1) {};
\node[item,right=0cm of item1] (item2) {};
\node[item,right=0cm of item2] (item3) {};
\node[item,right=0cm of item3] (item4) {};
\node[item,right=0cm of item4] (item5) {};
%
\node[transform,below=0.25cm of item1,minimum width=0.5cm] (map1) {};
\node[transform,below=0.25cm of item2,minimum width=0.5cm] (map2) {};
\node[transform,below=0.25cm of item3,minimum width=0.5cm] (map3) {};
\node[transform,below=0.25cm of item4,minimum width=0.5cm] (map4) {};
\node[transform,below=0.25cm of item5,minimum width=0.5cm] (map5) {};
%
\node[result,below=0.5cm of map5] (result) {result};
%
\path[arrow](item0) -- (map1);
\path[arrow](item1) -- (map1);
\path[arrow](map1) -- (map2);
\path[arrow](item2) -- (map2);
\path[arrow](map2) -- (map3);
\path[arrow](item3) -- (map3);
\path[arrow](map3) -- (map4);
\path[arrow](item4) -- (map4);
\path[arrow](map4) -- (map5);
\path[arrow](item5) -- (map5);
%
\path[arrow](map5) -- (result);
\end{tikzpicture}


%  \item Given:
%    \begin{itemize}
%      \item A sequence $x_1, x_2, \ldots, x_N \in T$.
%      \item An identity value $id \in I$.
%      \item A combine operation $c : I \times T \mapsto I$
%        \begin{itemize}
%          \item $c(c(x,y),z) \equiv c(x,c(y,z))$
%          \item $c(id,x) = \bar{x}$, where $\bar{x}$ is the value of $x$ in $I$.
%          \item $c(id,c(id,x)) = c(id,x)$
%          \item $c(c(c(id,x),y),c(c(id,z),t)) = c(c(c(c(id,x),y),z),t)$
%        \end{itemize}
%    \end{itemize}
%  \vfill
%  \item It generates the value:
%    \begin{itemize}
%      \item $c(\ldots c(c(id,x_1), x_2) \ldots, x_N)$
%    \end{itemize}
\end{itemize}
\end{frame}

\begin{frame}[t,fragile]{Homogeneous reduction}
\begin{block}{Add a sequence of values}
\begin{lstlisting}
template <typename Execution>
double add_sequence(const Execution & ex, const vector<double> & v)
{
  return grppi::reduce(ex, v, 0.0,
    [](double x, double y) { return x+y; });
}
\end{lstlisting}
\end{block}
\end{frame}

\begin{frame}[t,fragile]{Heterogeneous reduction}
\begin{block}{Add areas of shapes}
\begin{lstlisting}
template <typename Execution>
int add_areas(const Execution & ex, const std::vector<shape> & shapes)
{
  return grppi::reduce(ex, shapes, 0.0,
    [](double a, const auto & s) { 
        return a + s.area()(); 
    });
}
\end{lstlisting}
\end{block}
\begin{itemize}
  \item \textbad{Note}: Better expressed as a map reduce.
\end{itemize}
\end{frame}
