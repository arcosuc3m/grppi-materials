\begin{frame}[t]{Farm pattern}
\begin{itemize}
  \item A \textgood{farm} is a streaming pattern applicable to a stage in a \textmark{pipeline},
        providing multiple tasks to process data items from a data stream.
    \begin{itemize}
      \item A \textmark{farm} has an associated \textgood{cardinality} which is the number of parallel
             tasks used to serve the stage.
      \item Each task in a \textmark{farm} runs a \textgood{transformer} for each data item it receives.
    \end{itemize}
\end{itemize}
\vfill
\begin{center}
\begin{tikzpicture}
\tikzset{
  label/.style={text centered, text=orange,font=\footnotesize,minimum width=1cm} ,
  transform/.style={rectangle,rounded corners,draw=black,fill=green!50,text=white,thick, text centered, font=\tiny, minimum width=0.75cm,minimum height=0.5cm},
  item/.style={rectangle,draw=black,fill=orange!70,text=white,thick, text centered, font=\tiny, minimum width=0.75cm},
  result/.style={rectangle,draw=black,fill=blue!70,text=white,thick, text centered, font=\tiny, minimum width=0.75cm},
  arrow/.style={->,thick,draw=black,font=\tiny},
}  
\node[transform] (stage0) {};
\node[transform, right=1cm of stage0] (stage1) {};
\node[transform,right=1cm of stage1] (stage2a) {};
\node[transform,above=0.5cm of stage2a] (stage2b) {};
\node[transform,below=0.5cm of stage2a] (stage2c) {};
\node[transform,right=1cm of stage2a] (stage3) {};
\node[transform,right=1cm of stage3] (stage4) {};
%
\path[arrow] (stage0) -- (stage1);
\path[arrow] (stage1) -- (stage2a);
\path[arrow] (stage1) -- (stage2b);
\path[arrow] (stage1) -- (stage2c);
\path[arrow] (stage2a) -- (stage3);
\path[arrow] (stage2b) -- (stage3);
\path[arrow] (stage2c) -- (stage3);
\path[arrow] (stage3) -- (stage4);
\end{tikzpicture}
\end{center}
\end{frame}

\begin{frame}[t,fragile]{Farms in pipelines}
\begin{block}{Improving a video}
\begin{lstlisting}
template <typename Execution>
void run_pipe(const Execution & ex, std::ifstream & filein, std::ofstream & fileout)
{
  grppi::pipeline(ex,
    [&filein] () -> optional<frame> {
      frame f = read_frame(filein);
      if (!filein) retrun {};
      return f;
    },
    farm(4, [](const frame & f) { return improve(f); },
    [&fileout] (const frame & f) { write_frame(f); }
  );
}
\end{lstlisting}
\end{block}
\end{frame}

\begin{frame}[t,fragile]{Piecewise farms}
\begin{block}{Improving a video}
\begin{lstlisting}
template <typename Execution>
void run_pipe(const Execution & ex, std::ifstream & filein, std::ofstream & fileout)
{
  auto my_filter = farm(4, [](const frame & f) { return improve(f); });

  grppi::pipeline(ex,
    [&filein] () -> optional<frame> {
      frame f = read_frame(filein);
      if (!filein) retrun {};
      return f;
    },
    my_filter,
    [&fileout] (const frame & f) { write_frame(f); }
  );
}
\end{lstlisting}
\end{block}
\end{frame}
