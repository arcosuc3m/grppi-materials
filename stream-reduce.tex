\subsection{Reductions in pipelines}

\begin{frame}[t]{Stream reduction pattern}
\begin{itemize}
  \item A \textgood{stream reduction} pattern performs a reduction over the items of a
        subset of a data stream.
  \vfill\pause
  \item \textenum{Key elements}
    \begin{itemize}
      \item \textmark{window-size}: Number of elements in a reduction window.
      \item \textmark{offset}: Distance between two consecutive window starts.
      \item \textmark{identity}: Value used as identity in reductions.
      \item \textmark{combiner}: Combination operation used for reductions.
    \end{itemize}
\end{itemize}
\end{frame}

\begin{frame}[t,fragile]{Windowed reductions}
\begin{block}{Chunked sum}
\begin{lstlisting}
template <typename Execution>
void print_primes(const Execution & ex, int n)
{
  grppi::pipeline(exec,
    [i=0,max=n]() mutable -> optional<double> {
      if (i<=n) return i++;
      else return {};
    },
    grppi::reduce(100, 50, 0.0,
      [](double x, double y) { return x+y; }),
    [](int x) { cout << x << "\n"; }
  );
}
\end{lstlisting}
\end{block}
\end{frame}
