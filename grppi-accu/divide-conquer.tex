\begin{frame}[t]{Divide/conquer pattern}
\begin{itemize}
  \item A \textgood{divide/conquer} pattern splits a problem into two or more independent subproblems until a base case is reached.
    \begin{itemize}
      \item The base case is solved directly.
      \item The results of the subproblems are combined until the final solution of the original problem is obtained.
    \end{itemize}

  \vfill\pause
  \item \textenum{Key elements}:
    \begin{itemize}
      \item \textmark{Divider}: Divides a problem in a set of subproblems.
      \item \textmark{Solver}: Solves and individual subproblem.
      \item \textmark{Combiner}: Combines two solutions.
    \end{itemize}
\end{itemize}
\end{frame}

\begin{frame}[t,fragile]{A patterned merge/sort}
\begin{block}{Ranges on vectors}
\begin{lstlisting}
struct range {
  range(std::vector<double> & v) : first{v.begin()}, last{v.end()} {}
  auto size() const { return std::distance(first,last); }
  std::vector<double>::iterator first, last;
};

std::vector<range> divide(range r) {
  auto mid = r.first + r.size() / 2;
  return { {r.first, mid}, {mid, r.last} };
}
\end{lstlisting}
\end{block}
\end{frame}

\begin{frame}[t,fragile]{A patterned merge/sort}
\begin{block}{Ranges on vectors}
\begin{lstlisting}[escapechar=@]
template <typename Execution>
void merge_sort(const Execution & ex, std::vector<double> & v)
{
  grppi::divide_conquer(exec, range(v),
    // Divide range in sub-ranges
    [](auto r) -> vector<range> {
      if (1>=r.size()) return {r};
      else return divide(r);
    },@\pause@
    // A unit range is already ordered
    [](auto x) { return x; },@\pause@
    // Merge sorted subranges
    [](auto r1, auto r2) {
      std::inplace_merge(r1.first, r1.last, r2.last);
      return range{r1.first, r2.last};
    });
}
\end{lstlisting}
\end{block}
\end{frame}
