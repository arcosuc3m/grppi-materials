\subsection{GrPPI architecture}

\begin{frame}[t]{Some ideals}
\begin{itemize}[<+->]
  \item Applications should be expressed independently of the
        execution model.
  \item Multiple back-ends should be offered with simple switching
        mechanisms.
  \item Interface should integrate seamlessly with modern C++ standard
        library.
  \item Make use of modern (C++14) language features.
\end{itemize}
\end{frame}

\begin{frame}[t]{GrPPI}
\begin{Large}
\textgood{\url{https://github.com/arcosuc3m/grppi}}
\end{Large}
\vfill\pause
\begin{itemize}
  \item A header only library (might change).
  \item A set of execution policies.
  \item A set of type safe generic algorithms.
  \item Requires \textgood{C++14}.
  \item GNU GPL v3.
\end{itemize}
\end{frame}

\begin{frame}[t,fragile]{Setting up GrPPI}
\begin{itemize}
  \item \textenum{Structure}.
    \begin{itemize}
      \item \textmark{include}: Include files.
      \item \textmark{unit\_tests}: Unit tests using GoogleTest.
      \item \textmark{samples}: Sample programs.
      \item \textmark{cmake-modules}: Extra CMake scripts.
    \end{itemize}
  \vfill\pause
  \item Initial setup
\begin{lstlisting}[style=terminal]
mkdir build
cd build
cmake ..
make
\end{lstlisting}
\end{itemize}
\end{frame}

\begin{frame}[t]{CMake variables}
\begin{itemize}
  \item \textmark{GRPPI\_UNIT\_TESTS\_ENABLE}: Enable building unit tests.
  \item \textmark{GRPPI\_OMP\_ENABLE}: Enable OpenMP back-end.
  \item \textmark{GRPPI\_TBB\_ENABLE}: Enable Intel TBB back-end.
  \item \textmark{GRPPI\_EXAMPLE\_APPLICATIONS\_ENABLE}: Enable building example applications.
  \item \textmark{GRPPI\_DOXY\_ENABLE}: Enable documentation generation.
\end{itemize}
\end{frame}

\begin{frame}[t]{Execution policies}
\begin{itemize}
  \item The execution model is encapsulated by execution values.
  \vfill
  \item Current execution types:
    \begin{itemize}
      \item \cppid{sequential\_execution}.
      \item \cppid{parallel\_execution\_native}.
      \item \cppid{parallel\_execution\_omp}.
      \item \cppid{parallel\_execution\_tbb}.
      \item \cppid{dynamic\_execution}.
    \end{itemize}
  \vfill
  \item All top-level patterns take one \emph{execution} object.
\end{itemize}
\end{frame}

\begin{frame}[t]{Concurrency degree}
\begin{itemize}
  \item Sets the number of underlying threads used by the execution implementation.
    \begin{itemize}
      \item \cppid{sequential\_execution} $\Rightarrow$ 1
      \item \cppid{parallel\_execution\_native} $\Rightarrow$ \cppid{hardware\_concurrency()}.
      \item \cppid{parallel\_execution\_omp} $\Rightarrow$ \cppid{omp\_get\_num\_threads()}.
    \end{itemize}

  \vfill
  \item \textenum{API}
    \begin{itemize}
      \item \cppid{ex.set\_concurrency\_degree(4)}
      \item \cppkey{int }\cppid{n = ex.concurrency\_degree()}
    \end{itemize}
\end{itemize}
\end{frame}

\begin{frame}[t,fragile]{Dynamic back-end}
\begin{itemize}
  \item Useful if you want to take the decision at run-time.
  \item Holds any other execution policy (or empty).
\end{itemize}
\vfill\pause
\begin{block}{Selecting the execution back-end}
\begin{lstlisting}
grppi::dynamic_execution execution_mode(const std::string & opt) {
  using namespace grppi;
  if ("seq" == opt) return sequential_execution{};
  if ("thr" == opt) return parallel_execution_native{};
  if ("omp" == opt) return parallel_execution_omp{};
  if ("tbb" == opt) return parallel_execution_tbb{};
  return {};
}
\end{lstlisting}
\end{block}
\end{frame}

\begin{frame}[t]{Function objects}
\begin{itemize}
  \item GrPPI is heavily based on passing code sections as function objects
        (aka \emph{functors}).
  \vfill
  \item \textenum{Alternatives}:
    \begin{itemize}
      \item Standard C++ predefined functors (e.g. \cppid{std::plus<int>}).
      \item Custom hand-written function objects.
      \item Lambda expressions.
    \end{itemize}
  \vfill
  \item Usually lambda expressions lead to more concise code.
\end{itemize}
\end{frame}

